
Para calcular a probabilidade de erro $P(e)$ de cada constelação~\ref{eq:Pe_M_QAM} desenvolvida em~\cite{Cecilio}.
\begin{equation}
    P(e) = 4 \left(1-\frac{1}{\sqrt{M}}\right) Q\left(\sqrt{\frac{3}{M-1}\frac{E_s}{N_0}}\right) - 4\left(1-\frac{1}{\sqrt{M}}\right)^2 Q^2\left(\sqrt{\frac{3}{M-1}\frac{E_s}{N_0}}\right)
    \label{eq:Pe_M_QAM}
\end{equation}

Para valores mais elevados de \textit{SNR}, a equação da probabilidade do $M$-QAM pode ser reduzida para~\ref{eq:Pe_reduzida_M_QAM}, pois o segundo termo ao quadrado passa a ser irrelevante.
\begin{equation}
    P(e) = 4 \left(1-\frac{1}{\sqrt{M}}\right) Q\left(\sqrt{\frac{3}{M-1}\frac{E_s}{N_0}}\right)
    \label{eq:Pe_reduzida_M_QAM}
\end{equation}
