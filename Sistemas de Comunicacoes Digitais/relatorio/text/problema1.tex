\subsection{Problema 1 - $M$-QAM}

Considere a modulação $M$-QAM, em que o sinal em banda base é dado por:
$$s_m(t) = ( A_m^{(\text{real})} + j A_m^{(\text{imag})}) g(t) ,$$
em que $g(t)$ é um pulso transmitido, $A_m^{(\text{real})}$ e $A_m^{(\text{imag})}$ são amplitudes da parte real e imaginária da forma de onda transmitida, respectivamente.

Considere $\int_{-\inf}^{\inf} |g(t)|^2 \,dt = \mathcal{E}_{g} = 1$, isto é, o pulto $g(t)$ possui energia unitária. Suponha a transmissão de uma sequência de símbolo $\{s_{m}\}$ de tamanho $L = 26400 \text{bits}$
\begin{enumerate}
    \item Para $M = \{ 4, 16, 64\}$, determine a energia média $\mathcal{E}_{m}$ de cada constelação;
    \item Para $M = \{ 4, 16, 64\}$, determine a distância mínima $d_{min}$ entre dois símbolos;
    \item Para $M = \{ 4, 16, 64\}$, implemente o modulador (mapeamento bit-símbolo) usando a codificação de Gray;
    \item Para $M = \{ 4, 16, 64\}$, implemente o demodulador (mapeamento símbolo-bit).
\end{enumerate}

% Dica 1: Uma constelação $M$-QAM pode ser vista como um produto cartesiano de duas constelações $\sqrt{M}$-PAM.

% Dica 2: Para visualizar a constelação, pode-se usar a função \textit{scatter} do MATLAB.

% Dica 3: Você pode validar sua função de modulação e demodulação $M$-QAM, comparando a sequência original de bits gerada antes da modulação com aquela obtida após a demodulação. Se a implementação do modulador/demodulador estiver correta, ambas as sequências serão idênticas.

% Considere a modulação ̧$M$-QAM, em que o sinal em banda base ́e dado por:

% -------------------------------------------------------------------
\subsubsection{Energia da Constelação} 

O desenvolvimento é citado em~\cite{Proakis, Cecilio}.

\begin{table}[!ht]
    \centering
    \begin{tabular}{|c|c|c|c|}
    \hline
    $E_{media}$ & $E_{media(bit)}$ & $d_{min}$ \\ \hline
     &  &  \\ 
     $\frac{M-1}{3} \mathcal{E}_g$ & $ \frac{M-1}{3\log_2 M} \mathcal{E}_g$ & $\sqrt{\frac{6\log_2 M}{M-1} \mathcal{E}_{media(bit)}}$ \\ 
     &  &  \\ \hline
    \end{tabular}
    \caption{Frequência da onda de entrada e a tensão máxima da saída do circuito integrador.}
    \label{tab:QAM}
\end{table}

% -------------------------------------------------------------------
\subsubsection{Distância Mínima entre Símbolos}

A distância eucliadiana entre os sinais na modulação QAM é
$$ d_{mn} = \sqrt{|| s_m - s_n||^2}$$ 
$$ = \sqrt{\frac{\mathcal{E}_g}{2}[(A_{mi} - A_{ni})^2 + (A_{mq} - A_{nq})^2]}$$
\subsubsection{Modulador}

% -------------------------------------------------------------------
\subsubsection{Demodulador}

Considerando $\mathcal{E}_g = \int_{-\infty}^{\infty} |g(t)|^2 \,dt = 1$, a energia média da constelação pode ser calculada por $\epsilon$



%%
% Please add the following required packages to your document preamble:
% \usepackage[table,xcdraw]{xcolor}